%\documentclass[windows,csize4]{BHCexam}
\documentclass[windows,csize4,answers]{BHCexam}

\usepackage{multicol} % 分栏
\pagestyle{fancy}
\fancyfoot[C]{\kaishu \small 第 \thepage 页 共 \pageref{lastpage} 页}
%\fancyhead[L]{\includegraphics[width=2cm]{qrcode.png}}
\title{整式综合2作业}
%\subtitle{数学文科试卷}
%\notice{满分150分, 120分钟完成, \\	允许使用计算器,答案一律写在答题纸上.}
%\author{Gavin Chen}
%\date{\today}
\usepackage{enumerate} % 编号

\begin{document}
\maketitle


\begin{groups}
    \group{整式综合2}{}
    \begin{questions}[]

        \question[5] 已知$x-3y+4z=1,2x+y-2z=2$,化简$x^2-2xy-3y^2+2xz+10yz-8z^2$
        \begin{solution}{0.5cm}
            \methodonly 将$x^2-2xy-3y^2+2xz+10yz-8z^2$因式分解
            \[
                \begin{aligned}
                    &\phantom{=} x^2-2xy-3y^2+2xz+10yz-8z^2 \\ 
                    &=(x-3y+4z)(x+y-2z) \\ 
                    &=(x-3y+4z)(2x+y-2z-x) \\ 
                    &=2-x
                \end{aligned}
            \]
        \end{solution}
        \vspace{3.5cm}

        \question[5] 若$a,b,c$是整数,且$b$是正整数,同时他们满足$a+b=c,b+c=d,c+d=a$,那么$a+b+c+d$的最大值
        \begin{solution}{0.5cm}
            \methodonly 由$a+b=c, c+d=a$可得$b=-d$,代入$b+c=d$得到$c=2d, a=3d$。
            \[
                \begin{aligned}
                    &\phantom{=} a+b+c+d \\ 
                    &=3d+(-d)+2d+d \\ 
                    &=5d \\ 
                    &=-5b
                \end{aligned}
            \]
            因为$b$为正整数,所以最大值为$-5$。
        \end{solution}
        \vspace{3.5cm}

        \question[5] 若$a^2+2a+5$是$a^4+ma^2+n$的一个因式,求$mn$的值。
        \begin{solution}{0.5cm}
            \methodonly 由待定系数法,设$a^4+ma^2+n=(a^2+2a+5)(a^2+ua+\frac{n}{5})$
            \[
                \begin{cases}
                    u+2=0 \\
                    5+\frac{n}{5}+2u=m \\ 
                    \frac{2n}{5} +5u=0
                \end{cases}
            \]
            \[
                \begin{cases}
                    u=-2 \\
                    m=6 \\ 
                    n=25
                \end{cases}
            \]
        \end{solution}
        \vspace{3.5cm}

        \question[5] 若$a+b=10, a^3+b^3=100$,求$a^2+b^2$
        \begin{solution}{0.5cm}
            \methodonly $40$
        \end{solution}
        \vspace{3.5cm}

        \question[5] 若$a^3+b^3+c^3=a^2+b^2+c^2=a+b+c=1$,求$abc$的值。
        \begin{solution}{0.5cm}
            \methodonly
            \[
                \begin{aligned}
                    a+b+c&=1 \\ 
                    a^2+b^2+c^2+2ab+2bc+2ca&=1 \\ 
                    1+2(ab+bc+ca)&=1 \\ 
                    ab+bc+ca&=0
                \end{aligned}
            \]
            \[
                \begin{aligned}
                    a^3+b^3+c^3-3abc&=(a+b+c)(a^2+b^2+c^2-ab-bc-ca) \\ 
                    1-3abc&=1\cdot(1-0) \\ 
                    3abc&=0 \\
                    abc&=0
                \end{aligned}
            \]
        \end{solution}
        \vspace{3.5cm}

        \question[5] 若$x-y=1+m, y-z=1-m$,求$x^2+y^2+z^2-xy-yz-zx$
        \begin{solution}{0.5cm}
            \methodonly $x-y=1+m, y-z=1-m$得到$x-z=2$
            \[
                \begin{aligned}
                    &\phantom{=}x^2+y^2+z^2-xy-yz-zx \\ 
                    &=\frac{1}{2}[(x-y)^2+(y-z)^2+(z-x)^2] \\ 
                    &=m^2+3
                \end{aligned}
            \]
        \end{solution}
        \vspace{3.5cm}


        \question[5] 已知直角三角形的直角边为$a,b$,斜边为$c$。其中$a,b,c$均为整数并且$a$为质数。求证:
        $2(a+b+1)$是完全平方数。
        \begin{solution}{0.5cm}
            \methodonly 由直角三角形得$a^2=c^2-b^2=(c-b)(c+b)$,因为$a$是质数,故而
            \[
                \begin{cases}
                    c-b=1 \\
                    c+b=a^2
                \end{cases}
            \]
            得到$2b=a^2-1$,所以
            \[
                \begin{aligned}
                    &\phantom{=}2(a+b+1) \\
                    &=a^2+2a+1 \\ 
                    &=(a+1)^2
                \end{aligned}
            \]
            证毕。
        \end{solution}
        \vspace{3.5cm}

        \question[5] 已知$(1-ab)^2=(2ab-a-b)(a+b-2)$,求证$a,b$中至少有一个为$1$。
        \begin{solution}{0.5cm}
            \methodonly 令$a+b=x, ab=y$,原方程变为$(1-y)^2-(2y-x)(x-2)=0$。原方程左边
            \[
                \begin{aligned}
                    &\phantom{=}(1-y)^2-(2y-x)(x-2) \\
                    &=y^2-2y+1+x^2-2xy-2x+4y \\ 
                    &=y^2+2y+1+x^2-2x(y+1) \\ 
                    &=(y+1^2)-2x(y+1)+x^2 \\ 
                    &(y+1-x)^2
                \end{aligned}
            \]
            为$0$,即$y+1=x$,亦即$ab+1=a+b$,得到$(a-1)(b-1)=0$,故两者至少有一个为$1$。
        \end{solution}
        \vspace{3.5cm}

        \question[5] 若$x$是正整数,则$x^4-3x^2+9$是质数还是合数,并证明你的结论。
        \begin{solution}{0.5cm}
            \methodonly 
            \[
                \begin{aligned}
                    &\phantom{=}x^4-3x^2+9 \\
                    &=x^4+6x^2+9-9x^2 \\ 
                    &=(x^2+3)^2-9x^2 \\ 
                    &=(x^2+3x+3)(x^2-3x+3) 
                \end{aligned}
            \]
            当$x=1$,原式为$7$,质数。 \\ 
            当$x=2$,原式为$13$,质数。 \\ 
            当$x\ge 3$上述两个因式都大于$1$,所以$x\ge 3$时都是合数

        \end{solution}

    \end{questions}

\end{groups}


\label{lastpage}
\end{document}