\documentclass[windows,csize4]{BHCexam}
%\documentclass[windows,csize4,answers]{BHCexam}

\usepackage{multicol} % 分栏
\usepackage{polynom} % 多项式除法
\pagestyle{fancy}
\fancyfoot[C]{\kaishu \small 第 \thepage 页 共 \pageref{lastpage} 页}
%\fancyhead[L]{\includegraphics[width=2cm]{qrcode.png}}
\title{因式分解 - 对称式和轮换式}
%\subtitle{数学文科试卷}
%\notice{满分150分, 120分钟完成, \\	允许使用计算器,答案一律写在答题纸上.}
%\author{Gavin Chen}
%\date{\today}
\usepackage{enumerate} % 编号
\usepackage{cases}

\begin{document}

\maketitle

\begin{groups}
    \group{对称式和轮换式概念}{}
    对称式是代数中常常遇到的 一种特殊形式代数式。\\
    \fbox
    {
        \parbox{\textwidth}
        {
            在一个代数式中,如果把代数式中任意两个字母对换后代数式保持不变,则称这样的代数式
            为对称代数式,简称对称式。
        }
    }
    例如$a+b+c,x^2+2xy+y^2,\frac{1}{ab},a^3+b^3+c^3-3abc$等都是对称式,但是
    $a-b-c,\frac{1}{x-y},a+2b+3c$就不是对称式。 \\

    \fbox
    {
        \parbox{\textwidth}
        {
            在一个代数式中,如果把代数式中所含字母顺序轮换后代数式保持不变,
            则称这样的代数式为轮换对称代数式,简称轮换式。
        }
    }
    例如,$x^2y+y^2z+z^2x$ 中将$x$以$y$代换,$y$以$z$代换,$z$以$x$代换,可以得到
    $y^2z+z^2x+x^2y$,它和原式完全相同,所以它是关于$x,y,z$的轮换式。又比如$x^2+y^2+z^2$等等。 \\

    关于其次多项式的概念: \\
    \fbox
    {
        \parbox{\textwidth}
        {
            在一个代数式中,它所有项有相同的次数$n$,则称这样的多项式为$n$次齐次多项式。
        }
    }
    例如:$x^2+xy+y^2$是二次齐次多项式;$a^3+b^3+c^3-3abc$是三次齐次多项式。


    \group{对称式和轮换式性质}{}
    \begin{itemize}
        \item 对称式一定是轮换式,但是轮换式不一定是对称式。(想一想为什么?)\\
              例如:$x^2y+y^2z+z^2x$是轮换式,但不是对称式。
        \item 关于相同字母的对称式(轮换式)的和、差、积、商(假定除式不为零且能整除)仍然是对称式(轮换式)。\\
              例如:$a+b+c$和$abc$是关于$a,b,c$ 的对称式,他们的和$a+b+c+abc$,差$a+b+c-abc$,积$abc(a+b+c)$,商$\frac{a+b+c}{abc}$也都是对称式。
        \item 若对称式或者轮换式中含有某种形式的式子,则必定也含有这种形式的同形式。例如,关于$x,y,z$的二次齐次式中若含有$ax^2$,则必定含有$ay^2,az^2$。
              所以关于$x,y,z$的二次齐次对称式的一般形式是$a(x^2+y^2+z^2)+b(xy+yz+zx)$,其中$a,b$为系数。
    \end{itemize}

    一般性的,关于$x,y$的齐次对称式($a,b$为系数)是:
    \begin{itemize}
        \item 一次$a(x+y)$
        \item 二次$a(x^2+y^2)+bxy$
        \item 三次$a(x^3+y^3)+b(x^2y+y^2x)$
    \end{itemize}
    关于$x,y,z$的齐次对称式($a,b,c,d$为系数)是:
    \begin{itemize}
        \item 一次$a(x+y+z)$
        \item 二次$a(x^2+y^2+z^2)+b(xy+yz+zx)$
        \item 三次$a(x^3+y^3+z^3)+b(x^2y+y^2z+z^2x)+c(xy^2+yz^2+zx^2)+dxyz$
    \end{itemize}

    \group{对称式和轮换式因式分解的一般方法}{}
    因为对称多项式一定是轮换多项式,故而研究对称多项式的因式分解只需要研究轮换多项式因式分解就可以了。
    \begin{itemize}
        \item 选定一个字母(如$x$)做主元,其余看成常数。
        \item 利用因式定理确定它的因式,再利用轮换式性质,得到几个同类型的因式。
              例如,若$x-y$是轮换式的因式,那么$y-z$,$z-x$也一定是它的因式。
        \item 结合待定系数法求出系数。(通常可以只考虑特定几项或者采用特殊值的方法求解)
    \end{itemize}
    对于三个字母$x,y,z$的轮换多项式,常见因式有$xyz$,$(x+y+z)$,$(x+y)(y+z)(z+x)$,$(x-y)(y-z)(z-x)$,
    $(x+y-z)(y+z-x)(z+x-y)$等


    \group{对称式和轮换式例题1}{}
    \begin{questions}[]
        \question[5] 下面那些是轮换式,那些是对称式:\\
        $x+y+z$,$xyz$,$x^3+y^3+z^3-3xyz$,$x^2y+y^2x$,$x^2y+y^2z+z^2x$,$\frac{x^2}{x-1}+\frac{y^2}{y-z}$
        \begin{solution}{0.5cm}
            \methodonly 略。
        \end{solution}
        \vspace{3.5cm}

        \question[5] 分解因式$x^2(y-z)+y^2(z-x)+z^2(x-y)$
        \begin{solution}{0.5cm}
            \methodonly 显然这是一个关于$x,y,z$轮换式。
            把$x$作为主元,当$x=y$时,$f(y)=y^2(y-z)+y^2(z-y)+z^2(y-y)=0$,根据因式定理$x-y$是$f(x,y,z)$的一个因式。 \\
            由于它是轮换式,同理$y-z$和$z-x$也是它的因式。 故而$(x-y)(y-z)(z-x)$是它的因式。\\
            又由于原式最高次为三次,所以两者只相差一个常数项,即: \\
            \[
                f(x,y,z)= x^2(y-z)+y^2(z-x)+z^2(x-y) = k(x-y)(y-z)(z-x)
            \]
            此时由于是轮换式,我们只需要考虑某个特定项,而无需展开,比如仅考虑$x^2y$项,左边系数为$1$,右边系数为$-k$,故而可得$k=-1$。 \\
            所以
            \[
                x^2(y-z)+y^2(z-x)+z^2(x-y) = -(x-y)(y-z)(z-x)
            \]

        \end{solution}
        \vspace{3.5cm}

        \question[5] 分解因式$a^3(b-c)+b^3(c-a)+c^3(a-b)$
        \begin{solution}{0.5cm}
            \methodonly 显然这是一个关于$a,b,c$轮换式。与上一题类似,它有三个因式$(a-b)(b-c)(c-a)$ \\
            由于原式是四次多项式,而三个因式相乘只是三次多项式,两者差一次。考虑到它是轮换式,它已知因式也是轮换式,故而剩下的一次多项式也是轮换式。
            同时由于原式是四次齐次轮换式,它的一次因式也必然是齐次的(如果不是齐次,说明有常数项,那么相乘后必然有三次项)。所以只可能是$a+b+c$
            所以可得
            \[
                a^3(b-c)+b^3(c-a)+c^3(a-b)= k(a+b+c)(a-b)(b-c)(c-a)
            \]

            只考虑$a^3b$项,左边系数$1$,右边为$-ka^3b$,所以$k=-1$。

            当然们这里我们也可以换一种思考方式。既然等式恒成立,我们也可以选取特定的$a,b,c$来求解$k$。
            (注:选取特定值的时候得保证多项式不为0,不然就无法找出系数$k$了) \\
            这里可以选取$a=2,b=1,c=0$,代入后得到
            \[
                8-2+0=k\cdot 3  \cdot (-2)
            \]
            所以$k=-1$。注意这两种方法可以选取一种,甚至可以同时结合起来根据情况使用。
        \end{solution}
        \vspace{3.5cm}

        \question[5] 分解因式$x^3+y^3+z^3-3xyz$
        \begin{solution}{0.5cm}
            \methodonly 当$x=-(y+z)$的时候$f[-(x+y)]=-(y+z)^3+y^3+z^3+3(y+z)yz=0$
            所以原式含因式$x+y+z$,考虑到三次齐次多项式,可以设
            \[
                x^3+y^3+z^3-3xyz=(x+y+z)[a(x^2+y^2+z^2)+b(xy+yz+zx)]
            \]
            比较两边$x^3$项的系数,得到$a=1$,但是对于$b$则不太好处理。这时我们可以用特定值$x=y=z=1$ 代入,
            得到
            \[
                0=3\times (3+3b)
            \]
            所以$b=-1$,即$x^3+y^3+z^3-3xyz=(x+y+z)[(x^2+y^2+z^2)-(xy+yz+zx)]$

        \end{solution}
        \vspace{3.5cm}

        \question[5] 分解因式$(a+b+c)^3-(b+c-a)^3-(c+a-b)^3-(a+b-c)^3$
        \begin{solution}{0.5cm}
            \methodonly $a=0$时多项式为$0$,所以$a$是一个因式,同理$b,c$也是因式,即$abc$是它的因式。可以设
            \[
                (a+b+c)^3-(b+c-a)^3-(c+a-b)^3-(a+b-c)^3=kabc
            \]
            令$a=b=c=1$得到$k=24$,所以
            \[
                (a+b+c)^3-(b+c-a)^3-(c+a-b)^3-(a+b-c)^3=24abc
            \]
        \end{solution}
        \vspace{3.5cm}

        \question[5] 分解因式$(x-y)^5+(y-z)^5+(z-x)^5$
        \begin{solution}{0.5cm}
            \methodonly 容易得到$(x-y)(y-z)(z-x)$是它的因式,考虑到它是五次齐次多项式,所以可以设:
            \[
                (x-y)^5+(y-z)^5+(z-x)^5=(x-y)(y-z)(z-x)[l(x^2+y^2+z^2)+m(xy+yz+zx)]
            \]
            令$x=2,y=1,z=0$可以得到
            \begin{equation}
                \label{eq:sample_2_1}
                1-32+1=-2(5l+2m)
            \end{equation}
            令$x=1,y=0,z=-1$可以得到
            \begin{equation}
                \label{eq:sample_2_2}
                1-32+1=-2(2l-m)
            \end{equation}
            由\ref{eq:sample_2_1}式和\ref{eq:sample_2_2}式
            \begin{equation}
                \begin{cases}
                    5l+2m=15 \\
                    2l-m=15
                \end{cases}
            \end{equation}
            即:
            \begin{equation}
                \begin{cases}
                    l=5 \\
                    m=-5
                \end{cases}
            \end{equation}
            所以, 略 
        \end{solution}
        \vspace{3.5cm}

        \question[5] 分解因式$a^5-b^5-(a-b)^5$
        \begin{solution}{0.5cm}
            注意,这个不是轮换式。可以首先采取换元法。
            \methodonly 令$-b=c$原式可化为
            \[
                a^5+c^5-(a+c)^5
            \]
            显然当$a=0$或者$a=-c$时原式都为$0$。故而原式有有因式$ac(a+c)$。
            考虑到齐次五次式,故而可以设
            \[
                a^5+c^5-(a+c)^5=ac(a+c)[m(a^2+c^2)+nac]
            \]
            令$a=1,c=1$得到
            \begin{equation}
                \label{eq: eq_7_1}
                2-32=2\cdot (2m+n)
            \end{equation}
            令$a=2,c=1$得到
            \begin{equation}
                \label{eq: eq_7_2}
                32+1-243=2\cdot 3(5m+2n)
            \end{equation}  
            由 \ref{eq: eq_7_1}式和\ref{eq: eq_7_2}         
            \begin{equation}
                \begin{cases}
                    2m+n=-15 \\
                    5m+2n=-35
                \end{cases}
            \end{equation}
            \begin{equation}
                \begin{cases}
                    m=-5 \\
                    n=-5
                \end{cases}
            \end{equation}
            即$a^5+c^5-(a+c)^5=-5ac(a+c)[(a^2+c^2)+ac]$ \\
            将$c=-b$代回原式
            \[
                a^5-b^5-(a-b)^5=5ab(a-b)[(a^2+b^2)-ab]
            \]
        \end{solution}
        \vspace{3.5cm}

        \question[5] 分解因式$a^2b+ab^2+a^2c+ac^2+b^2c+bc^2+3abc$
        \begin{solution}{0.5cm}
            \method 原式时三次齐次多项式,当$a=-(b+c)$时,原式可化为
            \[
                \begin{aligned}
                    & \phantom{=} (b+c)^2b+[-(b+c)]b^2+(b+c)^2+[-(b+c)]c^2+b^2c+bc^2+3bc[-(b+c)] \\
                    &= (b+c)(b^2+bc-b^2+bc+c^2-c^2+bc-3bc) \\
                    &=0
                \end{aligned}
            \]
            故而有因式$a+b+c$,故而可设
            \[
                (a+b+c)[m(a^2+b^2+c^2)+n(ab+bc+ca)]  
            \]
            取$a=0,b=1,c=2$得$2=5m+2n$ \\ 
            取$a=0,b=2,c=2$得$1=2m+n$ \\
            最后得到$m=0,n=1$

            \method 此题完全可以用常规方法,反而更加简单
            \[
                \begin{aligned}
                    & \phantom{=} a^2b+ab^2+a^2c+ac^2+b^2c+bc^2+3abc \\
                    &= (a^2b+ab^2 +abc) +(a^2c+ac^2+abc)+(b^2c+bc^2+abc) \\
                    &= [ab(a+b+c)[ac(a+c+b)][bc(b+c+a)] \\
                    &= (ab+ac+bc)(a+b+c)
                \end{aligned}
            \]

        \end{solution}

    \end{questions}
\end{groups}



\label{lastpage}
\end{document}