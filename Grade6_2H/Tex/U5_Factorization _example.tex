\documentclass[windows,csize4]{BHCexam}
%\documentclass[windows,csize4,answers]{BHCexam}

\usepackage{multicol} % 分栏
\usepackage{polynom} % 多项式除法
\pagestyle{fancy}
\fancyfoot[C]{\kaishu \small 第 \thepage 页 共 \pageref{lastpage} 页}
%\fancyhead[L]{\includegraphics[width=2cm]{qrcode.png}}
\title{因式分解 - 对称式和轮换式}
%\subtitle{数学文科试卷}
%\notice{满分150分, 120分钟完成, \\	允许使用计算器,答案一律写在答题纸上.}
%\author{Gavin Chen}
%\date{\today}
\usepackage{enumerate} % 编号
\usepackage{cases}

\begin{document}

\maketitle

\begin{groups}
    \group{对称式和轮换式概念}{}
    对称式是代数中常常遇到的 一种特殊形式代数式。\\ 
    \fbox
    {
        \parbox{\textwidth}
        {
            在一个代数式中,如果把代数式中任意两个字母对换后代数式保持不变,则称这样的代数式
            为对称代数式,简称对称式。
        }
    }
    例如$a+b+c,x^2+2xy+y^2,\frac{1}{ab},a^3+b^3+c^3-3abc$等都是对称式,但是
    $a-b-c,\frac{1}{x-y},a+2b+3c$就不是对称式。 \\ 

    \fbox
    {
        \parbox{\textwidth}
        {
            在一个代数式中,如果把代数式中所含字母顺序轮换后代数式保持不变,
            则称这样的代数式为轮换对称代数式,简称轮换式。
        }
    }
    例如,$x^2y+y^2z+z^2x$ 中将$x$以$y$代换,$y$以$z$代换,$z$以$x$代换,可以得到
    $y^2z+z^2x+x^2y$,它和原式完全相同,所以它是关于$x,y,z$的轮换式。又比如$x^2+y^2+z^2$等等。 \\ 

    关于其次多项式的概念: \\
    \fbox
    {
        \parbox{\textwidth}
        {
            在一个代数式中,它所有项有相同的次数$n$,则称这样的多项式为$n$次齐次多项式。
        }
    }
    例如:$x^2+xy+y^2$是二次齐次多项式;$a^3+b^3+c^3-3abc$是三次齐次多项式。


    \group{对称式和轮换式性质}{}
    \begin{itemize}
        \item 对称式一定是轮换式,但是轮换式不一定是对称式。(想一想为什么?)\\ 
        例如:$x^2y+y^2z+z^2x$是轮换式,但不是对称式。
        \item 关于相同字母的对称式(轮换式)的和、差、积、商(假定除式不为零且能整除)仍然是对称式(轮换式)。\\ 
        例如:$a+b+c$和$abc$是关于$a,b,c$ 的对称式,他们的和$a+b+c+abc$,差$a+b+c-abc$,积$abc(a+b+c)$,商$\frac{a+b+c}{abc}$也都是对称式。
        \item to be continued
    \end{itemize}

    \group{对称式和轮换式例题}{}
    \begin{questions}[]
        \question[5] 若多项式
        \begin{solution}{0.5cm}
            \methodonly 由$m=5$。
        \end{solution}
        \vspace{3.5cm}


    \end{questions}
\end{groups}

\begin{groups}

\end{groups}
\group{待定系数法}{}
\fbox
{
    \parbox{\textwidth}
    {
        定理:一个整系数多项式如果能分解为两个有理系数的因式的乘积,那么也一定能分解为两个整系数的因式的积。
    }
}
根据以上定理,对于整系数的高次多项式因式分解,我们只需要讨论整系数的情况就可以了。
\begin{groups}
    \group{待定系数法例题}{}
    \begin{questions}

        \question[5] 分解因式:$x^4+x^3+2x^2-x+3$
        \begin{solution}{0.5cm}
            \methodonly 首先尝试试根法,经验证$\pm 1, \pm 3$都不满足条件$f(x)=0$的条件
            故而我们可以设
            \begin{equation}
                x^4+x^3+2x^2-x+3 = (x^2+ax+b)(x^2+cx+d) \label{eq:q_1}
            \end{equation}
            比较两边系数,可以得到
            \begin{numcases}{}
                a+c=1 \label{eq:q_2} \\
                b+d+ac=2 \label{eq:q_3} \\
                bc+ad=-1 \label{eq:q_4} \\
                bd=3 \label{eq:q_5}
            \end{numcases}
            这种方程组一般不易求解,但由于我们只需要考虑整系数的情况,那么
            \begin{numcases}{}
                b=1 \label{eq:q_6} \\
                d=3 \label{eq:q_7}
            \end{numcases}
            或者
            \begin{numcases}{}
                b=-1 \label{eq:q_6} \\
                d=-3 \label{eq:q_7}
            \end{numcases}
            注:$b=3,d=1$ 和$b=-3,d=-1$忽略,想一想为什么 \\
            将$b=1,d=3$代入后可得$a=-1,c=2$, 故而
            \[
                x^4+x^3+2x^2-x+3 = (x^2-x+1)(x^2+2x+3)
            \]
            若将$b=-11,d=-3$带入原方程组是矛盾的,舍去。故而最终答案唯一,即
            \[
                x^4+x^3+2x^2-x+3 = (x^2-x+1)(x^2+2x+3)
            \]
        \end{solution}
        \vspace{3.5cm}

        \question[5] 若$13x^3+mx^2+11x+n$能被$13x^2-6x+5$整除,求$m,n$的值.
        \begin{solution}{0.5cm}
            \methodonly 三次除以二次,商式最高为一次。可以设商式为$(x+a)$ 即
            \[
                \begin{aligned}
                     & \phantom{=}13x^3+mx^2+11x+n = (x+a)(13x^2-6x+5) \\
                     & =13x^3+(13a-6)x^2+(5-6a)x+5a
                \end{aligned}
            \]
            比较两边系数,可以得到
            \begin{numcases}{}
                13a-6=m \label{eq:eq2_1} \\
                5-6a=11 \label{eq:eq2_2} \\ 
                5a=n \label{eq:eq2_3}
            \end{numcases}
            \[
            \begin{cases}
                a=-1  \\
                m=-19  \\ 
                n=-5 
            \end{cases}
            \]
        \end{solution}
        \vspace{3.5cm}

        % 此题放最后一题
        \question[5] 已知$x^5-5qx+4r$有因式$(x-c)^2$,试说明$q^5=r^4$的理由.
        \begin{solution}{0.5cm}
            \methodonly 设
            \[
                x^5-5qx+4r=(x-c)^2(x^3+ax^2+bx+\frac{4r}{c^2})
            \]
            展开后比较系数
            \begin{numcases}{}
                a-2c=0 \label{eq:num1} \\
                c^2+b-2ac=0 \label{eq:num2} \\
                ac^2-2bc+\frac{4r}{c^2}=0 \label{eq:num3} \\
                \frac{8r}{c}-bc^2=5q \label{eq:num4}
            \end{numcases}
            由\ref{eq:num1}和\ref{eq:num2}可得$a=2c,b=3c^2$ \\
            代入\ref{eq:num3}和\ref{eq:num4}可得$r=c^5,q=c^4$ \\
            所以$q^5=r^4$
        \end{solution}

    \end{questions}
\end{groups}

\label{lastpage}
\end{document}