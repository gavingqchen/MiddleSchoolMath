\documentclass[windows,csize4]{BHCexam}
%\documentclass[windows,csize4,answers]{BHCexam}

\usepackage{multicol} % 分栏
\pagestyle{fancy}
\fancyfoot[C]{\kaishu \small 第 \thepage 页 共 \pageref{lastpage} 页}
%\fancyhead[L]{\includegraphics[width=2cm]{qrcode.png}}
\title{因式分解 - 对称式和轮换式作业}
%\subtitle{数学文科试卷}
%\notice{满分150分, 120分钟完成, \\	允许使用计算器,答案一律写在答题纸上.}
%\author{Gavin Chen}
%\date{\today}
\usepackage{enumerate} % 编号

\begin{document}
\maketitle


\begin{groups}
    \group{对称式和轮换式}{}
    \begin{questions}[]

        \question[5] 分解因式$(x-y)^3+(y-z)^3+(z-x)^3$
        \begin{solution}{0.5cm}
            \method $x=y$时为$f(x,y,z)=0$所以有因式$(x-y)(y-z)(z-x)$。又原式为三次齐次多项式,
            可以设
            \[
                f(x,y,z)=(x-y)^3+(y-z)^3+(z-x)^3=k(x-y)(y-z)(z-x)
            \]
            令$x=1, y=2, z=3$得到$k=3$,所以
            \[
                (x-y)^3+(y-z)^3+(z-x)^3=3(x-y)(y-z)(z-x)
            \]
            \method 直接欧拉公式
            \[
                (x-y)^3+(y-z)^3+(z-x)^3 - 3[(x-y)(y-z)(z-x)]= [(x-y)+(y-z)+(z-x)][\cdots]
            \]
            等式右边第一项为$0$,所以
            \[
                (x-y)^3+(y-z)^3+(z-x)^3 - 3[(x-y)(y-z)(z-x)]
            \]
            注意,对于欧拉公式
            \[
                a^3+b^3+c^3-3abc=(a+b+c)[a^2+b^2+c^2-ab-bc-ca]
            \]
            如果$a+b+c=0$, 那么$a^3+b^3+c^3=3abc$。此结论经常用到。
        \end{solution}
        \vspace{3.5cm}

        \question[5] 分解因式$(a+b+c)^4-(b+c)^4-(c+a)^4-(a+b)^4+a^4+b^4+c^4$
        \begin{solution}{0.5cm}
            \methodonly $a=0$时为$f(a,b,c)=0$所以有因式$abc$。又因为四次,可以设
            \[
                f(a,b,c)=(a+b+c)^4-(b+c)^4-(c+a)^4-(a+b)^4+a^4+b^4+c^4=kabc(a+b+c)
            \]
            令$a=b=c=1$得到$k=12$,所以
            \[
                (a+b+c)^4-(b+c)^4-(c+a)^4-(a+b)^4+a^4+b^4+c^4=12abc(a+b+c)
            \]
        \end{solution}
        \vspace{3.5cm}

        \question[5] 分解因式$(x+y+z)^3-x^3-y^3-z^3$
        \begin{solution}{0.5cm}
            \methodonly 当$x=-y$时,$f(x,y,z)=0$,故而$x+y$是它的因式,又由于轮换式,
            所以$(x+y)(y+z)(z+x)$是它的因式。考虑到原式为三次多项式,故而可设
            \[
                (x+y+z)^3-x^3-y^3-z^3=k(x+y)(y+z)(z+x)
            \]
            令$x=y=z=1$可得$k=3$,所以
            \[
                (x+y+z)^3-x^3-y^3-z^3=3(x+y)(y+z)(z+x)
            \]
        \end{solution}
        \vspace{3.5cm}

        \question[5] 分解因式$(a+b)^5-a^5-b^5$
        \begin{solution}{0.5cm}
            \methodonly $(a+b)^5-a^5-b^5=5ab(a+b)(a^2+b^2+ab)$
        \end{solution}
        \vspace{3.5cm}

        \question[5] 分解因式$(x+y+z)^5-x^5-y^5-z^5$
        \begin{solution}{0.5cm}
            \methodonly 当$x=-y$时原式为$0$,故原式含有因式$x+y$
            可以设$(x+y)(y+z)(z+x)[k(x^2+y^2+z^2)+l(xy+yz+zx)]$ 最后球的$k=l=5$
        \end{solution}
        \vspace{3.5cm}

        \question[5] 分解因式$a^4+b^4+c^4-2a^2b^2-2b^2c^2-2c^2a^2$
        \begin{solution}{0.5cm}
            \methodonly 当$a=\pm(b+c)$时原式为$0$,故原式可设
            \[
                k(a+b+c)(a-b-c)(b-c-a)(c-a-b)
            \]
            最后求得$k=1$
        \end{solution}
    \end{questions}

\end{groups}


\label{lastpage}
\end{document}