\documentclass[windows,csize4]{BHCexam}
%\documentclass[windows,csize4,answers]{BHCexam}

\usepackage{multicol} % 分栏
\pagestyle{fancy}
\fancyfoot[C]{\kaishu \small 第 \thepage 页 共 \pageref{lastpage} 页}
%\fancyhead[L]{\includegraphics[width=2cm]{qrcode.png}}
\title{整式的恒等变换1-例题}
%\subtitle{数学文科试卷}
%\notice{满分150分, 120分钟完成, \\	允许使用计算器,答案一律写在答题纸上.}
%\author{Gavin Chen}
%\date{\today}
\usepackage{enumerate} % 编号

\begin{document}
\maketitle


\begin{groups}

    \group{整式的恒等变换1-例题}{}
    \begin{questions}[]
        \question[5] 若$a,b,c$都是自然数,且满足$a^5=b^4,c^3=d^2$,并且$c-a=19$,求$d-b$的值
        \begin{solution}{0.5cm}
            \methodonly $757$
        \end{solution}
        \vspace{3.5cm}

        \question[5] 计算$(2+1)(2^2+1)(2^4+1)\cdots (2^{32}+1)+1$的值。
        \begin{solution}{0.5cm}
            \methodonly 
        \end{solution}
        \vspace{3.5cm}

        \question[5] 已知整数$a,b,(a-b)$都不是$3$的倍数,试证明$a^3+b^3$是$9$的倍数。
        \begin{solution}{0.5cm}
            \methodonly 提示,$a,b$不同余,设$a=3m+1,b=3n-1$然后带入
        \end{solution}
        \vspace{3.5cm}

        \question[5] 已知$a,b,c$为有理数,并且$a+b+c=0, a^3+b^3+c^3=0$。
        求证\quad 对于任何正奇数$n$,都有$a^n+b^n+c^n=0$。
        \begin{solution}{0.5cm}
            \methodonly 略
        \end{solution}
        \vspace{3.5cm}

        \question[5] 已知$a,b$为任意有理数,求证\quad 多项式$a^2+b^2-2a+6b+11$总是正数。
        \begin{solution}{0.5cm}
            \methodonly 略
        \end{solution}
        \vspace{3.5cm}

        \question[5] 若$14(a^2+b^2+c^2)=(a+2b+3c)^2$,求$a:b:c$。
        \begin{solution}{0.5cm}
            \methodonly 略
        \end{solution}
        \vspace{3.5cm}

        \question[5] 若$a,b,c,d$是整数,且$m=a^2+b^2,n=c^2+d^2$,求证\quad $mn$可以表示成两个整数的平方和。
        \begin{solution}{0.5cm}
            \methodonly 略
        \end{solution}
        \vspace{3.5cm}

        \question[5] 若$m,n$是自然数,且$m\neq n$,求证\quad 自然数$m^4+4n^4$一定可以表示成四个自然数的平方和。
        \begin{solution}{0.5cm}
            \methodonly 略
        \end{solution}
        \vspace{3.5cm}

        \question[5] 若$x+y=1,x^2+y^2=2$,求$x^7+y^7$的值。
        \begin{solution}{0.5cm}
            \methodonly 略
        \end{solution}
    \end{questions}

\end{groups}
\label{lastpage}
\end{document}