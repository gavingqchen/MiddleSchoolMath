%\documentclass[windows,csize4]{BHCexam}
\documentclass[windows,csize4,answers]{BHCexam}

\usepackage{multicol} % 分栏
\pagestyle{fancy}
\fancyfoot[C]{\kaishu \small 第 \thepage 页 共 \pageref{lastpage} 页}
%\fancyhead[L]{\includegraphics[width=2cm]{qrcode.png}}
\title{因式分解 - 换元法}
%\subtitle{数学文科试卷}
%\notice{满分150分, 120分钟完成, \\	允许使用计算器,答案一律写在答题纸上.}
%\author{Gavin Chen}
%\date{\today}
\usepackage{enumerate} % 编号


\begin{document}

\maketitle

\begin{groups}
    \group{换元法概念}{}
    “换元”就是“字母替代式”,即用新的“元”代替原来式中的式,使得原式编程含有新“元”
    的式子,然后对含有新元的式子按照要求求出结果,在将其代替的式子代回来,求出原式结果。
    \group{换元的作用}{}
    \begin{itemize}
        \item 为什么换元:化繁为简,比如高次化为低次,分式化成整式,等等。
        \item 什么时候换元:相同或者相似形式的代数式反复出现的时候。
        \item 最重要的,记得还原回来。
    \end{itemize}

    \group{常用换元方法}{}
    \begin{itemize}
        \item 局部换元
        \item 均值换元
        \item 和积换元
    \end{itemize}


\end{groups}

\begin{groups}
    \group{换元法例题}{}

    \begin{questions}[]
        \question[5]因式分解$(x^2+x-1)^2+(x^2+x-1)-2$
        \begin{solution}{0.5cm}
            \methodonly 设$x^2+x-1=u$,原式可化为
            \[
                \begin{aligned}
                     & \phantom{=}u^2+u-2   \\
                     & =(u-1)(u+2)          \\
                     & =(x^2+x-2)(x^2+x+1)  \\
                     & =(x-1)(x+2)(x^2+x+1)
                \end{aligned}
            \]
        \end{solution}
        \vspace{3.5cm}

        \question[5]因式分解$(x^2+4x+8)^2+3x(x^2+4x+8)+2x^2$ (局部换元)
        \begin{solution}{0.5cm}
            \methodonly 设$x^2+4x+8=u$,原式可化为
            \[
                \begin{aligned}
                     & \phantom{=}u^2+3xu+2x^2    \\
                     & =(u+x)(u+2x)               \\
                     & =(x^2+4x+8+x)(x^2+4x+8+2x) \\
                     & =(x^2+5x+8)(x^2+6x+8)      \\
                     & = (x+2)(x+4)(x^2+5x+8)
                \end{aligned}
            \]
        \end{solution}
        \vspace{3.5cm}

        \question[5]因式分解$x^4+(2y+4)x^2+y^2+4y+4$ (局部换元)
        \begin{solution}{0.5cm}
            \methodonly 解法一:设$y+2=u$,原式可化为
            \[
                \begin{aligned}
                     & \phantom{=}x^4+2ux^2+u^2 \\
                     & =(x^2+u)^2               \\
                     & =(x^2+y+2)^2             \\
                \end{aligned}
            \]
            解法二:此题也可以设$x^2+y=m$,原式可化为
            \[
                \begin{aligned}
                     & \phantom{=}x^4+2x^2 y+y^2+4x^2+4y+4 \\
                     & =(x^2+y)^2 + 4(x^2+y) +4            \\
                     & =m^2+4m+4                           \\
                     & =(x^2+y+2)^2
                \end{aligned}
            \]
            解法三:此题也可以设$x^2+2=n$,原式可化为
            \[
                \begin{aligned}
                     & \phantom{=}x^4+4x^2+4+2x^2 y+4y+y^2 \\
                     & =(x^2+2)^2+2y(x^2+2)+y^2            \\
                     & =n^2+2yn+y^2                        \\
                     & =(x^2+y+2)^2
                \end{aligned}
            \]
        \end{solution}
        \vspace{3.5cm}

        \question[5] 因式分解$(2a^2+2a+1)b+a(a+1)(b^2+1)$
        \begin{solution}{0.5cm}
            \methodonly 此题展开后合并不容易,可以将$a+1$看成整体。设$a+1=u$
            那么$(a^2+2a+1)=u^2$,所以原式可化为
            \[
                \begin{aligned}
                     & \phantom{=}(u^2+a^2)b+ay(b^2+1) \\
                     & =u^2b+a^2b+aub^2+au             \\
                     & =(u^2b+au)+(a^2+aub^2)          \\
                     & =(ub+a)(u+ab)                   \\
                     & =(a+b+ab)(a+1+ab)
                \end{aligned}
            \]
        \end{solution}



    \end{questions}
\end{groups}



\label{lastpage}
\end{document}