%\documentclass[windows,csize4]{BHCexam}
\documentclass[windows,csize4,answers]{BHCexam}

\usepackage{multicol} % 分栏
\pagestyle{fancy}
\fancyfoot[C]{\kaishu \small 第 \thepage 页 共 \pageref{lastpage} 页}
%\fancyhead[L]{\includegraphics[width=2cm]{qrcode.png}}
\title{因式分解 - 换元法}
%\subtitle{数学文科试卷}
%\notice{满分150分, 120分钟完成, \\	允许使用计算器,答案一律写在答题纸上.}
%\author{Gavin Chen}
%\date{\today}
\usepackage{enumerate} % 编号


\begin{document}

\maketitle

\begin{groups}
    \group{换元法概念}{}
    “换元”就是“字母替代式”,即用新的“元”代替原来式中的式,使得原式变成含有新“元”
    的式子,然后对含有新元的式子按照要求求出结果,在将其代替的式子代回来,求出原式结果。
    \group{换元的作用}{}
    \begin{itemize}
        \item 为什么换元:化繁为简,比如高次化为低次,分式化成整式,等等。
        \item 什么时候换元:相同或者相似形式的代数式反复出现的时候。
        \item 最重要的,记得还原回来。
    \end{itemize}

    \group{常用换元方法}{}
    \begin{itemize}
        \item 局部换元
        \item 均值换元
        \item 和积$x+y;xy$形式
        \item $x+\frac{1}{x}$形式
    \end{itemize}


\end{groups}

\begin{groups}
    \group{换元法例题}{}

    \begin{questions}[]
        \question[5]因式分解:$(x^2+x-1)^2+(x^2+x-1)-2$
        \begin{solution}{0.5cm}
            \methodonly 设$x^2+x-1=u$,原式可化为
            \[
                \begin{aligned}
                     & \phantom{=}u^2+u-2   \\
                     & =(u-1)(u+2)          \\
                     & =(x^2+x-2)(x^2+x+1)  \\
                     & =(x-1)(x+2)(x^2+x+1)
                \end{aligned}
            \]
        \end{solution}
        \vspace{3.5cm}

        \question[5]因式分解:$(x^2+4x+8)^2+3x(x^2+4x+8)+2x^2$ (局部换元)
        \begin{solution}{0.5cm}
            \methodonly 设$x^2+4x+8=u$,原式可化为
            \[
                \begin{aligned}
                     & \phantom{=}u^2+3xu+2x^2    \\
                     & =(u+x)(u+2x)               \\
                     & =(x^2+4x+8+x)(x^2+4x+8+2x) \\
                     & =(x^2+5x+8)(x^2+6x+8)      \\
                     & = (x+2)(x+4)(x^2+5x+8)
                \end{aligned}
            \]
        \end{solution}
        \vspace{3.5cm}

        \question[5]因式分解:$x^4+(2y+4)x^2+y^2+4y+4$ (局部换元)
        \begin{solution}{0.5cm}
            \method 设$y+2=u$,原式可化为
            \[
                \begin{aligned}
                     & \phantom{=}x^4+2ux^2+u^2 \\
                     & =(x^2+u)^2               \\
                     & =(x^2+y+2)^2             \\
                \end{aligned}
            \]
            \method 此题也可以设$x^2+y=m$,原式可化为
            \[
                \begin{aligned}
                     & \phantom{=}x^4+2x^2 y+y^2+4x^2+4y+4 \\
                     & =(x^2+y)^2 + 4(x^2+y) +4            \\
                     & =m^2+4m+4                           \\
                     & =(x^2+y+2)^2
                \end{aligned}
            \]
            \method 此题也可以设$x^2+2=n$,原式可化为
            \[
                \begin{aligned}
                     & \phantom{=}x^4+4x^2+4+2x^2 y+4y+y^2 \\
                     & =(x^2+2)^2+2y(x^2+2)+y^2            \\
                     & =n^2+2yn+y^2                        \\
                     & =(x^2+y+2)^2
                \end{aligned}
            \]
        \end{solution}
        \vspace{3.5cm}

        \question[5] 因式分解:$(2a^2+2a+1)b+a(a+1)(b^2+1)$
        \begin{solution}{0.5cm}
            \methodonly 此题展开后合并不容易,可以将$a+1$看成整体。设$a+1=u$
            那么$(a^2+2a+1)=u^2$,所以原式可化为
            \[
                \begin{aligned}
                     & \phantom{=}(u^2+a^2)b+ay(b^2+1) \\
                     & =u^2b+a^2b+aub^2+au             \\
                     & =(u^2b+au)+(a^2+aub^2)          \\
                     & =(ub+a)(u+ab)                   \\
                     & =(a+b+ab)(a+1+ab)
                \end{aligned}
            \]
        \end{solution}

        \question[5] 因式分解:$(x+1)^4+(x+3)^4-272$ (均值换元)
        \begin{solution}{0.5cm}
            \methodonly 选取括号内的均值为元,设$x+2=u$,原式可化为
            \[
                \begin{aligned}
                     & \phantom{=}(u-1)^4+(u+1)^4-272        \\
                     & = [(u^2+1)-2u]^2 + [(u^2+1)+2u]^2-272 \\
                     & = 2(u^4+6u^2+1)-272                   \\
                     & = 2(u^4+6u^2-135)                     \\
                     & = 2(u^2-9)(u^2+15)                    \\
                     & = 2(u+3)(u-3)(u^2+15)                 \\
                     & = 2(x-1)(x+5)(x^2+4x+19)
                \end{aligned}
            \]
        \end{solution}
        \vspace{3.5cm}

        \question[5] 因式分解:$(x+1)(x+3)(x+5)(x+7)+15$
        \begin{solution}{0.5cm}
            \method 一般会凑成$ax^2+bx+m$形式然后对常数取平均换元,注意这不是绝对的。取其他值(比如较小值)还原一样可行。
            \[
                \begin{aligned}
                     & \phantom{=}[(x+1)(x+7)][(x+3)(x+5)]+15 \\
                     & = (x^2+8x+7)(x^2+8x+15) +15
                \end{aligned}
            \]
            设$x^2+8x+11=u$,原式可化为
            \[
                \begin{aligned}
                     & \phantom{=}(u-4)(u+4)+15 \\
                     & =  u^2-16+15             \\
                     & = (u+1)(u-1)             \\
                     & = (x^2+8x+12)(x^2+8x+10) \\
                     & = (x+2)(x+6)(x^2+8x+10)
                \end{aligned}
            \]
            \method 设$x^2+8x+7=v$,原式可化为
            \[
                \begin{aligned}
                     & \phantom{=}v(v+8)+15     \\
                     & =  v^2+8v+15             \\
                     & = (v+3)(v+5)             \\
                     & = (x^2+8x+12)(x^2+8x+10) \\
                     & = (x+2)(x+6)(x^2+8x+10)
                \end{aligned}
            \]
        \end{solution}
        \vspace{3.5cm}

        \question[5] 因式分解:$(x^2+6x+8)(x^2+14x+48)+12$
        \begin{solution}{0.5cm}
            \methodonly 原式可化为
            \[
                \begin{aligned}
                     & \phantom{=}(x+2)(x+4)(x+6)(x+8)+12 \\
                     & = [(x+2)(x+8)][(x+4)(x+6)]+12      \\
                     & = (x^2+10x+16)(x^2+10x+24) + 12    \\
                \end{aligned}
            \]
            设$x^2+10x+20=m$,换元后可得
            \[
                \begin{aligned}
                     & \phantom{=}(m-4)(m+4)+12   \\
                     & = m^2-4                    \\
                     & = (m+2)(m-2)               \\
                     & = (x^2+10x+22)(x^2+10x+18)
                \end{aligned}
            \]
        \end{solution}
        \vspace{3.5cm}


        \question[5] 证明:四个连续整数的乘积加$1$是整数的平方。
        \begin{solution}{0.5cm}
            \methodonly 设四个整数分别为$x+1,x+2,x+3,x+4$ 那么
            \[
                \begin{aligned}
                     & \phantom{=}(x+1)(x+2)(x+3)(x+4)+1 \\
                     & = [(x+1)(x+4)][(x+2)(x+3)]+1      \\
                     & = (x^2+5x+4)(x^2+5x+6) +1
                \end{aligned}
            \]
            设$x^2+5x+4=t$ 上式可化为
            \[
                \begin{aligned}
                     & \phantom{=}t(t+2)+1 \\
                     & = (t+1)^2           \\
                     & = (x^2+5x+5)^2
                \end{aligned}
            \]
        \end{solution}
        \vspace{3.5cm}

        \question[5] 因式分解:$(x^2+3x+3)(x^2+7x+3)+3x^2$ (局部换元)
        \begin{solution}{0.5cm}
            \methodonly 设$x^2+3x+3=t$ 原式可化为
            \[
                \begin{aligned}
                     & \phantom{=}t(t+4x)+3x^2 \\
                     & = t^2+4xt+3x^2          \\
                     & = (t+x)(t+3x)
                     & = (x^2+4x+3)(x^2+6x+3)  \\
                     & = (x+1)(x+3)(x^2+6x+3)
                \end{aligned}
            \]
        \end{solution}
        \vspace{3.5cm}

        \question[5] 因式分解:$(x^2-x-2)(x^2+9x-2)+9x^2$ (局部换元)
        \begin{solution}{0.5cm}
            \methodonly 设$x^2-x-2=t$ 原式可化为 (注也可设$x^2+4x-2=m$求解)
            \[
                \begin{aligned}
                     & \phantom{=}t(t+10x)+9x^2 \\
                     & = t^2+10xt+9x^2          \\
                     & = (t+x)(t+9x)            \\
                     & = (x^2-2)(x^2+8x-2)
                \end{aligned}
            \]
        \end{solution}
        \vspace{3.5cm}

        \question[5] 因式分解:$(x+y-2xy)(x+y-2)+(1-xy)^2$ (积和换元)
        \begin{solution}{0.5cm}
            \methodonly 设$x+y=m,xy=n$ 原式可化为
            \[
                \begin{aligned}
                     & \phantom{=}(m-2n)(m-2)+(1-n)^2 \\
                     & = m^2-2m-2mn+4n+n^2-2n+1       \\
                     & = m^2-2mn+n^2-2m+2n+1          \\ %双十字
                     & = (m-n-1)^2                    \\
                     & = (x+y-xy-1)^2                 \\
                     & = (1-y)^2(x-1)^2
                \end{aligned}
            \]
        \end{solution}
        \vspace{3.5cm}

        \question[5] 因式分解:$x^4+x^3+\frac{9}{4} x^2+x+1$
        \begin{solution}{0.5cm}
            \methodonly 原式可化为
            \[
                \begin{aligned}
                     & \phantom{=}x^2\left(x^2+x+\frac{9}{4}+\frac{1}{x}+\frac{1}{x^2}\right) \\
                     & = x^2\left[\left(x^2+2+\frac{1}{x^2}\right)+
                        \left(x+\frac{1}{x}\right)+\frac{1}{4}\right]                         \\
                     & = x^2\left[\left(x+\frac{1}{x}\right)^2+
                        \left(x+\frac{1}{x}\right)+\frac{1}{4}\right]
                \end{aligned}
            \]
            设$x+\frac{1}{x}=t$,原式可转化为
            \[
                \begin{aligned}
                     & \phantom{=}x^2\left(t^2+t+\frac{1}{4}\right)              \\
                     & = x^2 \left(t+\frac{1}{2}\right)^2                        \\
                     & = \left[x \left(t+\frac{1}{2}\right)\right]^2             \\
                     & = \left[x \left(x+\frac{1}{x}+\frac{1}{2}\right)\right]^2 \\
                     & = \frac{1}{4}(2x^2+x+2)^2
                \end{aligned}
            \]
        \end{solution}

        \question[5] 因式分解:$(x+y+z)^3+(3x-2y-3z)^3-(4x-y-2z)^3$
        \begin{solution}{0.5cm}
            \methodonly 设$x+y+z=a;3x-2y-3z=b$,那么$4x-y-2z=a+b$,原式可化为
            \[
                \begin{aligned}
                     & \phantom{=}a^3+b^3-(a+b)^3 \\
                     & = (a+b)(a^2-ab+b^2)-(a+b)^3 \\ 
                     & = (a+b)[a^2-ab+b^2-(a+b)^2] \\ 
                     & = (a+b)(-3ab) \\ 
                     &-3(4x-y-2z)(x+y+z)(3x-2y-3z)
                \end{aligned}
            \]
        \end{solution}
        \vspace{3.5cm}

    \end{questions}
\end{groups}

\label{lastpage}
\end{document}