\documentclass[windows,csize4]{BHCexam}
%\documentclass[windows,csize4,answers]{BHCexam}

\usepackage{multicol} % 分栏
\usepackage{polynom} % 多项式除法
\pagestyle{fancy}
\fancyfoot[C]{\kaishu \small 第 \thepage 页 共 \pageref{lastpage} 页}
%\fancyhead[L]{\includegraphics[width=2cm]{qrcode.png}}
\title{因式分解 - 试根法,因式定理}
%\subtitle{数学文科试卷}
%\notice{满分150分, 120分钟完成, \\	允许使用计算器,答案一律写在答题纸上.}
%\author{Gavin Chen}
%\date{\today}
\usepackage{enumerate} % 编号


\begin{document}

\maketitle

\begin{groups}
    \group{整式乘法和除法}{}
    \begin{itemize}
        \item 整式乘法
              \begin{equation}
                \label{eq:sample}
                  (x+1)(2x^2+3x-2) = 2x^3+5x^2+x-2
              \end{equation}
        \item 整式除法 \\
              注意:由于\LaTeX \quad Polynom宏包的原因,多项式除法在竖式的表述和国内通常使用的长除法相差一个负号。故我们在长除竖式中的减法在这里需要用加法,下同。\\
              \polylongdiv{2x^3+5x^2+x-2 }{x+1} \\
              既$(2x^3+5x^2+x-2)\div (x+1)=(2x^2+3x-2)$
    \end{itemize}

    \group{因式定理}{}
    \fbox
    {
        \parbox{\textwidth}
        {
            因式定理:如果多项式$f(a)=0$,那么多项式$f(x)$必定含有因式$x-a$;
            反过来,如果$f(x)$含有因式$x-a$,那么$f(a)=0$。
        }
    }
    以\ref{eq:sample}式为例,$f(x)=2x^3+5x^2+x-2=(x+1)(2x^2+3x-2)$包含因式$(x+1)$,所以$f(-1)=0$;
    反过来$(-1)=0$,所以$f(x)=2x^3+5x^2+x-2=(x+1)=(2x^2+3x-2)$包含因式$(x+1)$

    \group{试根法}{}
    \fbox
    {
        \parbox{\textwidth}
        {
            试根法:分解高次多项式$f(x)$,用\uwave{常数项因数}和\uwave{最高次项因数}的比值(即为$a$)去试根,
            若验证$f(a)=0$则$x-a$可整除原多项式,即$(x-a)$为$f(x)$的因式。
        }
    }
    仍然以\ref{eq:sample}式为例,常数项$-2$有因数$\pm 1,\pm 2$,最高此项系数$2$有因数$\pm 1,\pm 2$,
    那么如果存在有理数$a$使得$f(a)=0$,则$a$只可能在$\pm 1, \pm 2, \pm\frac{1}{2}$中选取(注,三次一定有实数解哦)。 
    再看\ref{eq:sample}式, 完全因式分解后为$(x+1)(2x-1)(x+2)$,亦即有$f(-1)=0,f(\frac{1}{2})=0,f(-2)=0$


    \group{关于试根法系数关系的证明}{}


\end{groups}

\begin{groups}
    \group{试根法和因式定理例题}{}

    \begin{questions}[]
        \question[5]因式分解:$(x^2+x-1)^2+(x^2+x-1)-2$
        \begin{solution}{0.5cm}
            \methodonly 设$x^2+x-1=u$,原式可化为
            \[
                \begin{aligned}
                     & \phantom{=}u^2+u-2   \\
                     & =(u-1)(u+2)          \\
                     & =(x^2+x-2)(x^2+x+1)  \\
                     & =(x-1)(x+2)(x^2+x+1)
                \end{aligned}
            \]
        \end{solution}
        \vspace{4cm}


    \end{questions}
\end{groups}

\label{lastpage}
\end{document}