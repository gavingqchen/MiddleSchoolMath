%\documentclass[windows,csize4]{BHCexam}
\documentclass[windows,csize4,answers]{BHCexam}

\usepackage{multicol} % 分栏
\usepackage{polynom} % 多项式除法
\pagestyle{fancy}
\fancyfoot[C]{\kaishu \small 第 \thepage 页 共 \pageref{lastpage} 页}
%\fancyhead[L]{\includegraphics[width=2cm]{qrcode.png}}
\title{因式分解 - 试根法,因式定理}
%\subtitle{数学文科试卷}
%\notice{满分150分, 120分钟完成, \\	允许使用计算器,答案一律写在答题纸上.}
%\author{Gavin Chen}
%\date{\today}
\usepackage{enumerate} % 编号


\begin{document}

\maketitle

\begin{groups}
    \group{整式乘法和除法}{}
    \begin{itemize}
        \item 整式乘法
              \begin{equation}
                  \label{eq:sample}
                  (x+1)(2x^2+3x-2) = 2x^3+5x^2+x-2
              \end{equation}
        \item 整式除法 \\
              注:由于\LaTeX \quad Polynom宏包的原因,多项式除法在竖式的表述和国内通常使用的长除法相差一个负号。故我们在长除竖式中的减法在这里需要用加法,下同。\\
              \polylongdiv{2x^3+5x^2+x-2 }{x+1} \\
              既$(2x^3+5x^2+x-2)\div (x+1)=(2x^2+3x-2)$
    \end{itemize}

    \group{因式定理}{}
    \fbox
    {
        \parbox{\textwidth}
        {
            因式定理:如果多项式$f(a)=0$,那么多项式$f(x)$必定含有因式$x-a$;
            反过来,如果$f(x)$含有因式$x-a$,那么$f(a)=0$。
        }
    }
    以\ref{eq:sample}式为例,$f(x)=2x^3+5x^2+x-2=(x+1)(2x^2+3x-2)$包含因式$(x+1)$,所以$f(-1)=0$;
    反过来$(-1)=0$,所以$f(x)=2x^3+5x^2+x-2=(x+1)=(2x^2+3x-2)$包含因式$(x+1)$

    \group{试根法}{}
    \fbox
    {
        \parbox{\textwidth}
        {
            试根法:分解高次多项式$f(x)$,用\uwave{常数项因数}和\uwave{最高次项因数}的比值(即为$a$)去试根,
            若验证$f(a)=0$则$x-a$可整除原多项式,即$(x-a)$为$f(x)$的因式。
        }
    }
    仍然以\ref{eq:sample}式为例,常数项$-2$有因数$\pm 1,\pm 2$,最高此项系数$2$有因数$\pm 1,\pm 2$,
    那么如果存在有理数$a$使得$f(a)=0$,则$a$只可能在$\pm 1, \pm 2, \pm\frac{1}{2}$中选取。
    再看\ref{eq:sample}式, 完全因式分解后为$(x+1)(2x-1)(x+2)$,亦即有$f(-1)=0,f(\frac{1}{2})=0,f(-2)=0$。

    \group{关于试根法中根和系数关系的证明}{}
    设
    \begin{equation}
        \label{eq:fun2}
        f(x)=a_0+a_1 x+a_2 x^2+a_3 x^3 +\cdots +a_n x^n
        (a_n\neq 0, a_i\in \mathbb{Z}, i=0,1,2\cdots n)
    \end{equation}
    存在有理根
    \begin{equation}
        \label{eq:fun2_ans}
        c=\frac{q}{p}
    \end{equation}
    使得$f(c)=0$, 其中$p,q$是互质的整数。将\ref{eq:fun2_ans}带入\ref{eq:fun2}得到
    \begin{equation}
        \label{eq:fun3}
        a_0+a_1\frac{q}{p}+a_2\frac{q^2}{p^2}+\cdots +a_n\frac{q^n}{p^n} = 0
    \end{equation}
    两边都乘以$p^n$可以得到
    \begin{equation}
        \label{eq:fun4}
        a_0p^n+a_1qp^{n-1}+\cdots+ a_{n-1}q^{n-1}p + a_nq^n = 0
    \end{equation}
    对于等式右边$p$可以整除$0$,故而$p$可以整除等式左边的多项式。又因为其他项都含有因数$p$,只需要考虑
    最后一项$a_n q^n$。而又由于$p,q$互质,所以$p$可以整除$a_n$,亦即$p$是$a_n$的因数。\\
    同理可得$q$是$a_0$的因数。
\end{groups}


\begin{groups}
    \group{试根法和因式定理例题}{}

    \begin{questions}[]
        \question[5] 若多项式$x^2-mx+6$有一个因式是$x-3$,求$m$的值。
        \begin{solution}{0.5cm}
            \methodonly 由题意可知$x=3$时原多项式的值为$0$,故而$m=5$。
        \end{solution}
        \vspace{3.5cm}

        \question[5] 若多项式$x^3+ax^2+bx+10$有一个因式是$x^2-3x-10$,求$a,b$的值。
        \begin{solution}{0.5cm}
            \methodonly 由题意可知$x=5$和$x=-2$时原多项式的值为$0$,带入后可得关于$a,b$方程组
            解方程组可得$a=-4,b=-7$
        \end{solution}
        \vspace{3.5cm}

        \question[5] 因式分解:$x^3+x^2-10x-6$
        \begin{solution}{0.5cm}
            \methodonly 最高次项系数$1$,常数项$-6$,所以如果存在有理根,可能的值为
            $x=\pm 1, x=\pm 2, x=\pm 3,x=\pm 6$经验证$x=3$时多项式的值为$0$。\\
            \polylongdiv{x^3+x^2-10x-6}{x-3} \\
            所以结果为$(x-3)(x^2+4x+2)$
        \end{solution}
        \vspace{3.5cm}

        \question[5] 因式分解:$x^3+6x^2+11x+6$
        \begin{solution}{0.5cm}
            \methodonly $(x+1)(x+2)(x+3)$
        \end{solution}
        \vspace{3.5cm}

        \question[5] 因式分解:$2x^3-5x^2+5x-3$
        \begin{solution}{0.5cm}
            \methodonly $f(\frac{3}{2})=0$ 所以$x-\frac{3}{2}$是一个因式。
            为避免分数计算,乘以$2$后$2x-3$仍然是它的因式。
            最后可得$(2x-3)(x^2-x+1)$
        \end{solution}

        \question[5] 因式分解:$f(x)=6x^4+5x^3+3x^2-3x-2$
        \begin{solution}{0.5cm}
            \methodonly $a_0=-2, a_n=6$ 所以$f(x)$的有理根只可能为
            $\pm 1, \pm 2, \pm \frac{1}{2}, \pm \frac{1}{3}, \pm \frac{2}{3}, \pm \frac{1}{6}$ \\
            经检验$-\frac{1}{2}$是一个根,所以$2x+1$是$f(x)$的因式,可得 \\
            \[(2x+1)(3x^3+x^2+x-2)\] \\
            对$3x^3+x^2+x-2$来说$\frac{2}{3}$是一个根,再一次用试根法最后可得 \\
            \[(2x+1)(3x-2)(x^2+x+1)\]
        \end{solution}

        \question[5] 系数为字母的情况因式分解:$x^3-(a+b+c)x^2+(ab+bc+ca)x-abc$
        \begin{solution}{0.5cm}
            \methodonly 常数项为$-abc$故而可能的因数为
            \[\pm a, \pm b, \pm c, \pm ab, \pm bc, \pm ca, \pm abc \]
            经验证$a$是一个根,即$x-a$是一个因式
            \[
                \begin{aligned}
                     & \phantom{=} x^3-(a+b+c)x^2+(ab+bc+ca)x-abc \\
                     & =(x^3-ax^2)-[(b+c)x^2-a(b+c)x]+(bcx-abc)   \\
                     & =(x-a)[x^2-(b+c)x+bc]                      \\
                     & =(x-a)(x-b)(x-c)
                \end{aligned}
            \]
        \end{solution}

        \question[5] 因式分解:$(l+m)x^3+(3l+2m-n)x^2+(2l-m-3n)x-2(m+n)$
        \begin{solution}{0.5cm}
            \methodonly 当多项式所有项的系数相加和为$0$,那么$1$一定是它的根;当多项式
            偶次项的系数的和减去奇次项系数的和等于$0$,那么$-1$一定是它的根。\uwave{想一想为什么?}
            $-(l+m)+(3l+2m-n)-(2l-m-3n)-2(m+n)=0$ \\ 
            用多项式长除法可得
            \[
                (x+1)[(l+m)x^2+(2l+m-n)x-2(m+n)]
            \]
            然后对$(l+m)x^2+(2l+m-n)x-2(m+n)$十字相乘 \\ 
            \[
                (x+1)(x+2[(l+m)x-(m+n)])
            \]

        \end{solution}

        tobe continued
        余式定理简介, 
        待定系数

    \end{questions}
\end{groups}

\label{lastpage}
\end{document}