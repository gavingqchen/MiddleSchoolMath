\documentclass[windows,csize4]{BHCexam}
%\documentclass[windows,csize4,answers]{BHCexam}

\usepackage{multicol} % 分栏
\pagestyle{fancy}
\fancyfoot[C]{\kaishu \small 第 \thepage 页 共 \pageref{lastpage} 页}
%\fancyhead[L]{\includegraphics[width=2cm]{qrcode.png}}
\title{分式的概念和化简求值-作业}
%\subtitle{数学文科试卷}
%\notice{满分150分, 120分钟完成, \\	允许使用计算器,答案一律写在答题纸上.}
%\author{Gavin Chen}
%\date{\today}
\usepackage{enumerate} % 编号

\begin{document}
\maketitle


\begin{groups}
    \group{分式的概念和化简求值}{}
    \begin{questions}[]

        \question[5] 已知$\frac{1}{n}-\frac{1}{m}-\frac{1}{m-n}=0$,求$\left(\frac{m}{n}+\frac{n}{m}\right)^2$的值。
        \begin{solution}{0.5cm}
            \methodonly 由已知$(m-n)^2=mn$,所以$m^2+n^2=3mn$,所以原式
            \[
                \begin{aligned}
                    &\phantom{=}\left(\frac{m}{n}+\frac{n}{m}\right)^2 \\
                    &= \left(\frac{m^2+n^2}{mn}\right)^2 \\ 
                    &=9
                \end{aligned}
            \]
        \end{solution}
        \vspace{4.5cm}

        \question[5] 已知$\frac{1}{x}=\frac{2}{y+z}=\frac{3}{z+x}$,求$\frac{3z+5y}{4x}$的值。
        \begin{solution}{0.5cm}
            \methodonly 此题倒过来设$k$。\\
            设$\frac{x}{1}=\frac{y+z}{2}=\frac{z+x}{3}=k$,则
            \[
            \begin{cases}
                x=k \\
                y+z=2k \\
                z+x=3k
            \end{cases}
            \]
            即
            \[
            \begin{cases}
                x=k \\
                y=0k \\
                z=2k
            \end{cases}
            \]
            所以$\frac{3z+5y}{4x}=\frac{6k}{4k}=\frac{3}{2}$。
        \end{solution}
        \vspace{4.5cm}

        \question[5] 已知$x^2-8x+13=0$,求$\frac{x^4-6x^3-2x^2+18x+23}{x^2-8x+15}$
        \begin{solution}{0.5cm}
            \methodonly 分子除以$x^2-8x+15$后余$10$,分母除以$x^2-8x+15$后余$2$,所以为$5$。
        \end{solution}
        \vspace{4.5cm}

        \question[5] 已知$x+y+z=3$,且$x,y,z$不全相等,则$\frac{3(x-1)(y-1)(z-1)}{(x-1)^3+(y-1)^3+(z-1)^3}$
        \begin{solution}{0.5cm}
            \methodonly 换元$x-1=u, y-1=v, z-1=w$后欧拉公式,结果为$1$
        \end{solution}
        \vspace{4.5cm}

        \question[5] 已知$x^2-3x+1=0$,求$\frac{2x^5-5x^4+2x^3-8x^2}{x^2+1}$的值。
        \begin{solution}{0.5cm}
            \methodonly 直接分子去凑比较麻烦,可以直接多项式除法。\\
            将分子除以$x^2-3x+1$,\\
            $2x^5-5x^4+2x^3-8x^2=(x^2-3x+1)(2x^3+x^2+3x)-3x$,即原式可化为 \\
            $\frac{-3x}{x^2+1}=\frac{-(x^2+1)}{x^2+1}=-1$
        \end{solution}
        \vspace{4.5cm}

        \question[5] 已知$a+b+c=0$,且$a,b,c$均不为$0$,求$\frac{1}{b^2+c^2-a^2}+\frac{1}{c^2+a^2-b^2}+\frac{1}{a^2+b^2-c^2}$的值。
        \begin{solution}{0.5cm}
            \methodonly 由$a+b=-c$得到$a^2+b^2-c^2=-2ab$,同理$b^2+c^2-a^2=-abc, c^2+a^2-b^2=-2ca$原式结果为$0$。
        \end{solution}
        \vspace{4.5cm}

        \question[5] 已知$x,y,z$为三个不相等的非零实数,且$x+\frac{1}{y}=y+\frac{1}{z}=z+\frac{1}{x}$,求证:$x^2y^2z^2=1$。
        \begin{solution}{0.5cm}
            \methodonly 由$x+\frac{1}{y}=y+\frac{1}{z}$得到$x-y=\frac{y-z}{yz}$,由因为$x\neq y$所以$yz=\frac{y-z}{x-y}$。\\
            同理$xy=\frac{x-y}{z-x},\quad zx=\frac{z-x}{y-z}$。\\
            相乘后可得$x^2y^2z^2=1$。
        \end{solution}
        \vspace{4.5cm}

        \question[5] 有理数$x,y,z$满足$(y-z)^2+(z-x)^2+(x-y)^2=(y+z-2x)^2+(z+x-2y)^2+(x+y-2z)^2$,且$y^2\neq 1$。
        求$\frac{(yz+1)(x^2-1)}{(zx+1)(y^2-1)}$的值。
        \begin{solution}{0.5cm}
            \methodonly 设$x-y=a, \quad y-z=b, \quad z-x=c$,代入原式展开可得$a^2+b^2+c^2-2ab-2bc-2ca=0$。\\
            又$a+b+c=0$,显然$a^2+b^2+c^2+2ab+2bc+2ca=0$.\\
            两式相加$a^2+b^2+c^2=0$,即$a=b=c=0, \quad x=y=z$。于是所求值为$1$。
        \end{solution}

    \end{questions}

\end{groups}


\label{lastpage}
\end{document}