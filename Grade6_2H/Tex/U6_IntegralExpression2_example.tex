%\documentclass[windows,csize4]{BHCexam}
\documentclass[windows,csize4,answers]{BHCexam}

\usepackage{multicol} % 分栏
\usepackage{polynom} % 多项式除法
\pagestyle{fancy}
\fancyfoot[C]{\kaishu \small 第 \thepage 页 共 \pageref{lastpage} 页}
%\fancyhead[L]{\includegraphics[width=2cm]{qrcode.png}}
\title{整式综合2}
%\subtitle{数学文科试卷}
%\notice{满分150分, 120分钟完成, \\	允许使用计算器,答案一律写在答题纸上.}
%\author{Gavin Chen}
%\date{\today}
\usepackage{enumerate} % 编号
\usepackage{cases}

\begin{document}

\maketitle

\begin{groups}

    \group{对整式综合2}{}
    \begin{questions}[]

        \question[5]  在三角形$ABC$中,$a^2-16b^2-c^2+6ab+10bc=0$,其中$a,b,c$是三角形的三个边,求证:$a+c=2b$
        \begin{solution}{0.5cm}
            \methodonly 证明:将原等式的左边部分用双十字相乘进行因式分解(注意$ac$项系数为$0$) \\
            \[
                \begin{aligned}
                     & \phantom{=} a^2-16b^2-c^2+6ab+10bc \\
                     & = (a+8b-c)(a-2b+c)
                \end{aligned}
            \]
            当然也可以采用主元法因式分解
            \[
                \begin{aligned}
                     & \phantom{=} a^2-16b^2-c^2+6ab+10bc \\
                     & = a^2+6ab-(8b-c)(2b-c)             \\
                     & =[a+(8b-c)[a-(ab-c)]               \\
                     & = (a+8b-c)(a-2b+c)
                \end{aligned}
            \]
            所以可以得到$a+8b-c=0$或者 $a-2b+c=0$。\\
            考虑到$a,b,c$是三角形三边,而三角形两边之和大于第三边,故而$a+8b-c\neq 0$,所以可以得到$a+c=2b$.
        \end{solution}
        \vspace{3.5cm}

        \question[5]  已知$a^2b+ac^2+b^2c=b^2a+bc^2+a^2c$
        \begin{solution}{0.5cm}
            \method
            \[
                \begin{aligned}
                     & a^2b+ac^2+b^2c=b^2a+bc^2+a^2c,         \\
                     & (a^2b-a^2c)+(b^2c-c^2b)+(c^2a-b^2a)=0, \\
                     & a^2(b-c)+bc(b-c)-a(b+c)(b-c)=0,        \\
                     & (b-c)[(a^2-ab)+(bc-ac)]=0,             \\
                     & (b-c)(a-b)(a-c) =0
                \end{aligned}
            \]
            \method 将原方程可化为
            \[
                a^2b+ac^2+b^2c-(b^2a+bc^2+a^2c)=0
            \]
            其等式左边是一个轮换式,且有因式$a-b$,故而也有因式$b-c$和$c-a$,即
            \[
                a^2b+ac^2+b^2c-(b^2a+bc^2+a^2c)=(a-b)(b-c)(c-a)=0
            \]
        \end{solution}
        \vspace{3.5cm}


        \question[5]  解方程$(16x+27)^2(8x+15)(2x+3)=7$
        \begin{solution}{0.5cm}
            \methodonly 方程两边都乘以$16$
            \[
                \begin{aligned}
                    (16x+27)^2(8x+15)(2x+3)                 & =7         \\
                    (16x+27)^2[2\cdot(8x+15)][8\cdot(2x+3)] & =16\cdot 7 \\
                    (16x+27)^2(16x+30)(16x+24)              & =112
                \end{aligned}
            \]
            换元法设$16x+27=t$得到
            \[
                \begin{aligned}
                    t^2(t+3)(t-3)   & =112 \\
                    t^4-9t^2-112    & =0   \\
                    (t^2-16)(t^2+7) & =0
                \end{aligned}
            \]
            由于$t^2\ge 0$, 故而$t^2=16$,即$t=\pm 4$ \\
            还原成$x$,得到$x=-\frac{23}{16}$或者$x=-\frac{31}{16}$
        \end{solution}
        \vspace{3.5cm}

        \question[5] 已知$a,b,c,d$满足 $a+b=c+d$,$a^3+b^3=c^3+d^3$,求证:$a^{2021}+b^{2021}=c^{2021}+d^{2021}$
        \begin{solution}{0.5cm}
            \methodonly 显然不是直接求值的。
            由$a+b=c+d$ 可以得到$(a+b)^3=(c+d)^3$故而
            \begin{equation}
                \label{eq:5_1}
                a^3+3a^2b+3ab^2+b^3=c^3+3c^2d+3cd^2+d^3
            \end{equation}
            减去$a^3+b^3=c^3+d^3$得到
            \begin{equation}
                \label{eq:5_2}
                ab(a+b)=cd(c+d)
            \end{equation}
            分类讨论: \\
            若$a+b=c+d=0$,则$a=-b,c=-d$,显然
            \[
                a^{2021}+b^{2021}=c^{2021}+d^{2021}=0
            \]
            若$a+b=c+d\neq 0$由\ref{eq:5_2}式可得
            \begin{equation}
                \label{eq:5_3}
                ab=cd
            \end{equation}
            从而
            \[
                (a-b)^2=(a+b)^2-4ab=(c+d)^2-4cd=(c-d)^2
            \]
            故而
            \begin{equation}
                \label{eq:5_4}
                a-b=c-d
            \end{equation}
            或者
            \begin{equation}
                \label{eq:5_5}
                a-b=d-c
            \end{equation}
            若\ref{eq:5_4}成立,在和$a+b=c+d$相加得到$a=c,b=d$; \\
            若\ref{eq:5_5}成立,在和$a+b=c+d$相加得到$a=d,b=c$。两者都容易得到
            \[
                a^{2021}+b^{2021}=c^{2021}+d^{2021}
            \]
            证毕
        \end{solution}
        \vspace{3.5cm}







        \question[5]  $a$
        \begin{solution}{0.5cm}
            \methodonly 略。
        \end{solution}
        \vspace{3.5cm}



    \end{questions}
\end{groups}



\label{lastpage}
\end{document}