\documentclass[windows,csize4]{BHCexam}
%\documentclass[windows,csize4,answers]{BHCexam}

\usepackage{multicol} % 分栏
\usepackage{polynom} % 多项式除法
\pagestyle{fancy}
\fancyfoot[C]{\kaishu \small 第 \thepage 页 共 \pageref{lastpage} 页}
%\fancyhead[L]{\includegraphics[width=2cm]{qrcode.png}}
\title{分式的概念和化简求值}
%\subtitle{数学文科试卷}
%\notice{满分150分, 120分钟完成, \\	允许使用计算器,答案一律写在答题纸上.}
%\author{Gavin Chen}
%\date{\today}
\usepackage{enumerate} % 编号
\usepackage{cases}

\begin{document}

\maketitle

\begin{groups}

    \group{分式的概念}

    关于分式的基本概念:\\
    \fbox
    {
        \parbox{\textwidth}
        {
            当两个整式相除,即 $A\div B$时,可以表示为$\frac{A}{B}$.如果$B$含有字母,那么$\frac{A}{B}$叫做分式。其中$A$为
            分式的分子,$B$为分式的分母。\uwave{如果分式的分母为零,那么这个分式无意义。}
        }
    }
    \group{分式的加减乘除和基本性质}

    分式加减乘除规则
    \begin{itemize}
        \item 分式加法 \\
              \[
                  \frac{a}{b}+\frac{c}{d}=\frac{ad}{bd}+\frac{cb}{bd}=\frac{ad+cb}{bd}
              \]

        \item 分式减法 \\
              \[
                  \frac{a}{b}-\frac{c}{d}=\frac{ad}{bd}-\frac{cb}{bd}=\frac{ad-cb}{bd}
              \]
        \item 分式乘法 \\
              \[
                  \frac{a}{b}\cdot\frac{c}{d}=\frac{ac}{bd}
              \]
        \item 分式乘法 \\
              \[
                  \frac{a}{b}\div\frac{c}{d}=\frac{a}{b}\cdot\frac{d}{c}=\frac{ad}{bc}
              \]
        \item 分式乘方 \\
              \[
                  \left(\frac{a}{b}\right)^n=\frac{a^n}{b^n}
              \]
    \end{itemize}

    分式基本性质
    \begin{itemize}
        \item
              \[
                  \frac{a}{b}=\frac{am}{bm}, \quad
                  \frac{a}{b}=\frac{a\div m}{b\div m} (m\neq 0)
              \]
        \item 分子、分母和分式本身的符号三者中,改变任何两个的符号,分式的值不变。
    \end{itemize}

    比例的性质
    \begin{itemize}
        \item 如果$\frac{a}{b}=\frac{c}{d}, \frac{c}{d}=\frac{e}{f}$,那么$\frac{a}{b}=\frac{e}{f}$。
        \item 如果$\frac{a}{b}=\frac{c}{d}$,那么$ad=bc$。
        \item 如果$\frac{a}{b}=\frac{c}{d}$,那么$\frac{a+b}{b}=\frac{c+d}{d}$(合比性质)。
        \item 如果$\frac{a}{b}=\frac{c}{d},(a-b\neq 0)$,那么$\frac{a+b}{a-b}=\frac{c+d}{c-d}$(合分比性质)。
        \item 如果$\frac{a}{b}=\frac{c}{d}=\cdots =\frac{m}{n}$,且$b+d+\cdots +n\neq 0$,
              那么$\frac{a+c+\cdots +m}{b+d+\cdots +n}=\frac{a}{b}$(等比性质)。
    \end{itemize}


    \group{例题}

    \begin{questions}[]

        \question[5] 若$a,b,c$为非零常数,且$\frac{a+b-c}{c}=\frac{a-b+c}{b}=\frac{-a+b+c}{a}$,
        求$\frac{(a+b)(b+c)(c+a)}{abc}$
        \begin{solution}{0.5cm}
            \methodonly 由
            \[
                \frac{a+b-c}{c}=\frac{a-b+c}{b}=\frac{-a+b+c}{a}
            \]
            得到
            \[
                \frac{a+b}{c}-1=\frac{a+c}{b}-1=\frac{b+c}{a}-1
            \]
            所以设
            \[
                \frac{a+b}{c}=\frac{a+c}{b}=\frac{b+c}{a}=k
            \]
            将三式相加$2(a+b+c)=k(a+b+c)$ \\
            当$a+b+c\neq 0$时,$k=2$\\
            当$a+b+c=0$时,$k=-1$\\
            \[
                \begin{aligned}
                     & \phantom{=} \frac{(a+b)(b+c)(c+a)}{abc} \\
                     & = \frac{k^3 abc}{abc}                   \\
                     & = k^3
                \end{aligned}
            \]
            所以原式等于$-1$或$8$。
        \end{solution}
        \vspace{4.5cm}

        \question[5]  已知$\frac{x}{a}+\frac{y}{b}+\frac{z}{c}=1, \frac{a}{x}+\frac{b}{y}+\frac{c}{z}=0$,
        求$\frac{x^2}{a^2}+\frac{y^2}{b^2}+\frac{z^2}{c^2}$的值。
        \begin{solution}{0.5cm}
            \method 把$\frac{x}{a}+\frac{y}{b}+\frac{z}{c}=1$两边平方,得
            \[
                \frac{x^2}{a^2}+\frac{y^2}{b^2}+\frac{z^2}{c^2}+2\left( \frac{xy}{ab}+\frac{yz}{bc}+\frac{zx}{ca} \right)=1
            \]
            整理得到
            \[
                \frac{x^2}{a^2}+\frac{y^2}{b^2}+\frac{z^2}{c^2}+2\frac{xyz}{abc} \left( \frac{c}{z}+\frac{a}{x}+\frac{b}{y} \right)=1
            \]
            又因为$\frac{a}{x}+\frac{b}{y}+\frac{c}{z}=0$,所以
            \[
                \frac{x}{a}+\frac{y}{b}+\frac{z}{c}=1
            \]
            \method 换元,设$\frac{x}{a}=u, \frac{y}{b}=v, \frac{z}{c}=w$,则原式可化为
            \[
                u+v+w=1,
                \frac{1}{u}+\frac{1}{v}+\frac{1}{w}=0
            \]
            由后一个条件通分$\frac{uv+vw+wu}{uvw}=0$,而$u,v,w$均不为零,得到$uv+vw+wu=0$。\\
            把$u+v+w=1$两边平方,得到$u^2+v^2+w^2+2(uv+vw+wu)=1$,于是$u^2+v^2+w^2=1$,
            即$\frac{x^2}{a^2}+\frac{y^2}{b^2}+\frac{z^2}{c^2}=1$
        \end{solution}
        \vspace{4.5cm}

        \question[5]  已知$\frac{x}{x^2+x+1}=\frac{1}{4}$,求$\frac{x^2}{x^4+x^2+1}$的值。
        \begin{solution}{0.5cm}
            \methodonly 由已知$\frac{x}{x^2+x+1}=\frac{1}{4}$可以得到
            \[
                \frac{x^2+x+1}{x}=4
            \]
            即
            \[
                x+\frac{1}{x}+1=4
            \]
            即$x+\frac{1}{x}=3$,$x^2+\frac{1}{x^2}=7$,所以
            \[
                \frac{x^4+x^2+1}{x^2}=x^2+\frac{1}{x^2}+1=7+1=8
            \]
            即$\frac{x^2}{x^4+x^2+1}=\frac{1}{8}$。
        \end{solution}
        \vspace{5cm}

        \question[5]  已知三个不全为$0$的数$x,y,z$满足$4x-3y-6z=0, x+2y-7z=0$。求$\frac{2x^2+3y^2+6z^2}{x^2+5y^2+7z^2}$的值
        \begin{solution}{0.5cm}
            \methodonly 两个方程三个未知数无法直接求解,但可以把其中一个未知数作为参数。由已知:
            \begin{numcases}{}
                4x-3y-6z=0 \label{eq:4_1} \\
                x+2y-7z=0  \label{eq:4_2}
            \end{numcases}
            (\ref{eq:4_2})$ \times 4 -$(\ref{eq:4_1})得到$y=2z$,\\
            代入(\ref{eq:4_2})得到$x=3z$。
            所以
            \[
                \frac{2x^2+3y^2+6z^2}{x^2+5y^2+7z^2}=\frac{2(3z)^2+3(2z)^2+6z^2}{(3z)^2+5(2z)^2+7z^2}=1
            \]

        \end{solution}
        \vspace{4.5cm}

        \question[5]  已知$x,y,z$为有理数,且$(x-y)^2+(y-z)^2+(z-x)^2=(y+z-2x)^2+(z+x-2y)^2+(x+y-2z)^2$,求
        $\frac{(yz+1)(zx+1)(xy+1)}{(x^2+1)(y^2+1)(z^2+1)}$的值。
        \begin{solution}{0.5cm}
            \methodonly 将已知条件展开得到$2x^2+2y^2+2z^2-2xy-2yz-2zx=0$,即$(x-y)^2+(y-z)^2+(z-x)^2=0$.所以$x=y=z$,故而原式$=1$。
        \end{solution}
        \vspace{4.5cm}

        \question[5]  已知$a\neq b, a\neq 0, b\neq 0, a+b\neq 0, x=\frac{4ab}{a+b}$,求$\frac{x+2a}{x-2a}+\frac{x+2b}{x-2b}$的值。
        \begin{solution}{0.5cm}
            \methodonly 直接代入计算比较复杂,但可以用合分比性质。
            由已知等式可得
            \[
                \frac{x}{2a}=\frac{2b}{a+b}, \quad \frac{x}{2b}=\frac{2a}{a+b}.    
            \]
            利用合分比性质
            \[
                \frac{x+2a}{x-2a}=\frac{2b+a+b}{2b-(a+b)}=\frac{3b+a}{b-a} 
            \]
            \[
                \frac{x+2b}{x-2b}=\frac{2a+a+b}{2a-(a+b)}=\frac{3a+b}{a-b}    
            \]
            所以
            \[
                \begin{aligned}
                    & \phantom{=}\frac{x+2a}{x-2a}+\frac{x+2b}{x-2b} \\
                    &= \frac{3b+a}{b-a} +\frac{3a+b}{a-b} \\
                    &= \frac{2a-2b}{a-b} \\ 
                    &= 2
                \end{aligned}
            \]

        \end{solution}
        \vspace{4.5cm}


        \question[5]  若$\frac{a+b-c}{c}=\frac{a-b+c}{b}=\frac{-a+b+c}{a}$,求$\frac{(a+b)(b+c)(c+a)}{abc}$的值。
        \begin{solution}{0.5cm}
            \methodonly 本题可用等比性质求解. \\
            若$a+b+c\neq 0$根据等比性质
            \[
                \frac{a+b-c}{c}=\frac{a-b+c}{b}=\frac{-a+b+c}{a}=\frac{(a+b-c)+(a-b+c)+(-a+b+c)}{a+b+c}=1
            \]
            所以$a+b=2c, b+c=2a, c+a=2b$, 代入后原式$=8$。\\
            若$a+b+c=0$,\\
            则$a+b=-c, b+c=-a, c+a=-b$,代入后原式$=-1$。
        \end{solution}

    \end{questions}
\end{groups}



\label{lastpage}
\end{document}