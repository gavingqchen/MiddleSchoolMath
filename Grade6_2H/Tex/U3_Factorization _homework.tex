%\documentclass[windows,csize4]{BHCexam}
\documentclass[windows,csize4,answers]{BHCexam}

\usepackage{multicol} % 分栏
\pagestyle{fancy}
\fancyfoot[C]{\kaishu \small 第 \thepage 页 共 \pageref{lastpage} 页}
%\fancyhead[L]{\includegraphics[width=2cm]{qrcode.png}}
\title{因式分解 - 换元法作业}
%\subtitle{数学文科试卷}
%\notice{满分150分, 120分钟完成, \\	允许使用计算器,答案一律写在答题纸上.}
%\author{Gavin Chen}
%\date{\today}
\usepackage{enumerate} % 编号

\begin{document}
\maketitle


\begin{groups}
    \group{因式分解}{}
    \begin{questions}[]

        \question[5] 因式分解:$(x^2+x+1)(x^2+x+2)-12$
        \begin{solution}{0.5cm}
            \methodonly $(x^2+x-2)(x^2+x+5)$
        \end{solution}
        \vspace{3.5cm}

        \question[5] 因式分解:$12(x+y)^2+11(x+y)(x-y)+2(x-y)^2$
        \begin{solution}{0.5cm}
            \methodonly 设$x+y=m,x-y=n$,原式可化为
            \[
                \begin{aligned}
                     & \phantom{=}12m^2+11mn+2n^2 \\
                     & = (3m+2n)(4m+n)            \\
                     & = (5x+y)(5x+3y)
                \end{aligned}
            \]
        \end{solution}
        \vspace{3.5cm}

        \question[5] 因式分解:$(x-1)x(x+1)(x+2)-24$
        \begin{solution}{0.5cm}
            \methodonly
            \[
                \begin{aligned}
                     & \phantom{=}(x-1)x(x+1)(x+2)-24 \\
                     & = [(x-1)(x+2)][x(x+1)]-24      \\
                     & = (x^2+x-2)(x^2+x)-24
                \end{aligned}
            \]
            设$x^2+x-1=u$ 原式可继续化为
            \[
                \begin{aligned}
                     & \phantom{=}(u-1)(u+1)-24 \\
                     & = u^2-25                 \\
                     & = (u+5)(u-5)             \\
                     & = (x^2+x+4)(x^2+x-6)     \\
                     & = (x+3)(x-2)(x^2+x+4)
                \end{aligned}
            \]
        \end{solution}
        \vspace{3.5cm}

        \question[5] 因式分解:$(x^2-15x+54)(x^2+11x+28)+350$
        \begin{solution}{0.5cm}
            \methodonly 原式可化为
            \[
                \begin{aligned}
                     & \phantom{=}(x-6)(x-9)(x+4)(x+7)+350 \\
                     & = [(x-6)(x+4)][(x-9)(x+7)] + 350    \\
                     & = (x^2-2x-24)(x^2-2x-63) +350
                \end{aligned}
            \]
            设$x^2-2x-36=u$ 原式可化为 (注意,这里设主元的时候选了相对中间的一个,为了计算方便)
            \[
                \begin{aligned}
                     & \phantom{=}(u+12)(u-27)+350 \\
                     & = u^2-15u-324 + 350         \\
                     & = u^2-15u+26                \\
                     & = (u-2)(u-13)               \\
                     & = (x^2-2x-38)(x^2-2x-49)
                \end{aligned}
            \]
        \end{solution}
        \vspace{3.5cm}

        \question[5] 因式分解:$(x^2+4x+6)(x^2+6x+6)-3x^2$
        \begin{solution}{0.5cm}
            \methodonly 设$x^2+4x+6=m$ 原式可化为
            \[
                \begin{aligned}
                     & \phantom{=}m\cdot (m+2x) -3x^2 \\
                     & = m^2+2xm-3x^2                 \\
                     & = (m+3x)(m-x)                  \\
                     & = (x^2+7x+6)(x^2+3x+6)         \\
                     & = (x+1)(x+6)(x^2+3x+6)
                \end{aligned}
            \]
        \end{solution}
        \vspace{3.5cm}

        \question[5] 因式分解:$(x^4-4x^2+1)(x^4+3x^2+1)+10x^4$
        \begin{solution}{0.5cm}
            \methodonly 设$x^4-4x^2+1=m$ 原式可化为
            \[
                \begin{aligned}
                     & \phantom{=}m\cdot (m+7x^2)+10x^4   \\
                     & = 10x^4 + 7mx^2 + m^2              \\
                     & = (5x^2+m)(2x^2+m)                 \\
                     & = (x^4 + x^2+1)(x^4-2x^2+1)        \\
                     & = (x^4+2x^2+1-x^2)(x+1)^2(x-1)^2   \\
                     & = [(x^2+1)^2 -x^2](x+1)^2(x-1)^2   \\
                     & = (x^2+x+1)(x^2-x+1)(x+1)^2(x-1)^2
                \end{aligned}
            \]
        \end{solution}
        \vspace{3.5cm}


        \question[5] 因式分解:$(x+y)^3+2xy(1-x-y)-1$
        \begin{solution}{0.5cm}
            \methodonly 设$x+y=m,xy=n$ 原式可化为
            \[
                \begin{aligned}
                     & \phantom{=}m^3+2n(1-m)-1         \\
                     & = m^3-2nm+2n-1                   \\
                     & = (m-1)(m^2+m+1) -2n(m-1)        \\
                     & = (m-1)(m^2+m-2n+1)              \\
                     & = (x+y-1)(x^2+y^2+2xy+x+y-2xy+1) \\
                     & = (x+y-1)(x^2+y^2+x+y+1)
                \end{aligned}
            \]
        \end{solution}
        \vspace{3.5cm}

        \question[5] $a$
        \begin{solution}{0.5cm}
            \methodonly $略$
        \end{solution}
        \vspace{3.5cm}


    \end{questions}

\end{groups}


\label{lastpage}
\end{document}